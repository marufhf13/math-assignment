\documentclass{article}
\usepackage{graphicx}
\usepackage{titling}
\usepackage{tikz}
\usepackage{geometry}
\usepackage{amsmath}
\usetikzlibrary{calc}

\geometry{a4paper, margin=1in}

\title{1st assignment}
\author{Maruf Hasan}
\date{\today}

\begin{document}

\begin{titlepage}
    \begin{tikzpicture}[remember picture,overlay]
        \draw[line width = 2pt]
            ($(current page.north west) + (2cm,-2cm)$) rectangle
            ($(current page.south east) + (-2cm,2cm)$);
    \end{tikzpicture}
    
    \begin{center}
        \vspace{2cm}
        \includegraphics[width=0.3\textwidth]{IU logo.jpg}\par
        \vspace{0.5cm}
        {\huge\textbf{Department of CSE}}\par
        \vspace{1cm}
        {\Large\textbf{Course Title:} Geometry and Vector Analysis\par}
        
        {\Large\textbf{Course Code:}  MATH-2109\par}
        \vspace{1cm}
        {\huge\textbf{Assignment on Chapter 1}\par}
        \vspace{2cm}
        \Large\textbf{Submitted by:}\par
        \textbf{Name:} Maruf Hasan\par
        \textbf{Roll No:} 2114024\par
        \textbf{Reg No:} 1089\par
        \textbf{Session:} 2021-2022\par
        \vspace{2cm}
        \Large\textbf{Submission Date:} October 20, 2024\par
    \end{center}
\end{titlepage}

\newpage

\begin{flushleft}


\textbf{1. Find the areas of triangles whose vertices are}\par
\textbf{(a) $(-3, 4)$, $(6, 2)$, $(4, -3)$}\par
\textbf{(b) $(a, b+c)$, $(b, c+a)$, $(c, a+b)$}\par
\vspace{0.2cm}
\textbf{2. If three points $(-1, 2), (2, -1),(h,3)$ are collinear ; show that $h=-2$}\par
\vspace{0.2cm}
\textbf{3. If the area of quadriatel, whose angular points $A, B, C, D$ taken in order are $(1,2),(-5,6),(7,-4),(k,-2)$ be zero find the value of $k$.}\par
\vspace{0.2cm}
\textbf{4. Show that three points $(4,2),(7,5),(9,4)$ lie on a right line.}\par
\vspace{0.2cm}
\textbf{5. If the area of the quadrilateral formed by the points $(1,3),(2,-5),(6,-2)$ and $(5,k)$ taken in order be $30$ , show that $k=4$ . }\par
\vspace{0.2cm}
\textbf{6. Find the coordinates of the orthocentre of the triangle whose points are $(1,0),(2,-4)$ and $(-5,-2)$ .}\par
\vspace{0.2cm}
\textbf{7. If the points $(a,b),(a',b')$ and $(a-a' , b-b')$ are colinear, show that their joint passes through the origin and that $ab'=ab'$ .}\par
\vspace{0.2cm}
\textbf{8. In what ratio is the straight line joining the pts. $(3,4)$ and $(8,1)$ divides by the $X$ axis. Find the abscissa of this point on $X$ axis.}\par
\vspace{0.2cm}
\textbf{9. Find the polar co-ordinates of the point whose cartesian co-ordinates are}\par
\textbf{(a) $(-\sqrt{3},1)$  }\par
\textbf{(b) $(5,12)$}\par
\vspace{0.2cm}
\textbf{10. Find the cartesian co-ordinates of the points whose polar co-ordinates are}\par
\textbf{(a) $(5, -\frac{\pi}{4})$}\par
\textbf{(b) $(2, 330^{\circ})$}\par
\vspace{0.2cm}
\textbf{11. Find the polar distance between the points whose polar co-ordinates are }\par
\textbf{(a) $(-3, 45^{\circ})$ and $(7, 105^{\circ})$}\par
\textbf{(b) $(\sqrt{2}, \frac{5\pi}{4} )$ and $(2, \frac{2\pi}{3})$}\par
\vspace{0.2cm}
\textbf{12. Find the area of triangle , the polar co-ordinates of whose angular points are $(a , \theta), (2a , \theta+\frac{\pi}{3}), (3a , \theta+\frac{2\pi}{3})$}\par
\vspace{0.2cm}
\textbf{13. Change the equations to polar co-ordinates}\par
\textbf{(a) $(x^2+y^2)^2 = 2a^2xy$}\par
\textbf{(b) $x^4 + x^2y^2 - (x+y)^2 = 0$}\par
\vspace{0.2cm}
\textbf{14. Transform the co-ordinates to equations}\par
\textbf{(a) $r^2-2r(cos\hspace{0.1cm}\theta-sin\hspace{0.1cm}\theta)-7 = 0$}\par
\textbf{(b) $r(1- e\hspace{0.1cm} cos\hspace{0.1cm}\theta) = e\hspace{0.1cm}p$}\par
\vspace{0.2cm}
\textbf{15. Find the equations to lines passing through $(-5,6)$ and (a) parallel (b) perpendicular to $7x-8y = 9$.}\par
\vspace{0.2cm}
\textbf{16. Find the area of triangle formed by the lines $2x=y-3 = 0$, $3x+2y-1 = 0$ and $2x+3y+4 = 0$.}\par
\vspace{0.2cm}
\textbf{17. Find the equations of the line which passes through the insertion of the lines $3x-5y+9 = 0$ and $4x +7y-28 = 0$ and satisfies the following conditions}\par
\textbf{(a) passes through $(4,2)$}\par
\textbf{(b) Whose intercepts are equal. }\par
\vspace{0.2cm}
\textbf{18. What are the pts on the $X$ axis whose perpendicular distance from the straight line $\frac{x}{a} + \frac{y}{b} = 1$ is $a$. }\par
\vspace{0.2cm}
\newpage
\textbf{19. If $p$ and $p_1$ be the perpendiculars from the origin upon the st lines whose equations are $x\hspace{0.1cm} sec\hspace{0.1cm}\theta+y \hspace{0.1cm}cosec\hspace{0.1cm} \theta = a$ and $x\hspace{0.1cm} cos\hspace{0.1cm}\theta - y \hspace{0.1cm}sin \hspace{0.1cm}\theta = a \hspace{0.1cm}cos\hspace{0.1cm} 2\theta $. Prove that $4p^2 + (p_1)^2 = a^2$.}\par
\vspace{0.2cm}
\textbf{20. The equation to a pair of opposite sides of a parallelogram are $x^2 -7x -6 = 0$ and $y^2 -11y  +40 =0$. Find the equations to its diagonals.}\par
\textbf{21. Prove that the line $x+\sqrt{3}y = 0$ bisects the $\Delta$ whose vertices are $(3,-\sqrt{3})$, $(-1,-\sqrt{3})$ and $(1,\sqrt{3})$.}\par
\vspace{0.2cm}
\textbf{22. Find the equation of the bisector of that angle between the lines $4x-3y+1 =0 $ and $12x - 5y +13 =0 $ in which the origin lies. }\par
\vspace{0.2cm}
\textbf{23. Find the equation of two st. lines passing through the points $(1,-1)$ and inclined with an angle $45^\circ$ with the st. line }\par
\vspace{0.2cm}
\textbf{24. A straight line moves so that the sum of the receiprocals of its intercepts on the axes is constant, show that it passes through a fixed point.}\par
\vspace{0.2cm}
\textbf{25. Find the area of the triangles formed by the lines $8x+7y-43=0$, $2x=5y = 1$ and $4x-3y+11 = 0$.}\par

\newpage
\textbf{1. \hspace{0.4cm}(a)}\par
\vspace{0.5cm}
 Here given : 
\[ x_1 = -3, y_1 = 4 \]
\[ x_2 = 6, y_2 = 2 \]
\[ x_3 = 4, y_3 = -3 \]
\[\text{Area} = \frac{1}{2} |(x_1y_2-x_2y_1)+(x_2y_3-x_3y_2)+(x_3y_1-x_1y_3)| \]
\[=\frac{1}{2} |(-6-24)+(-18-8)+(16-9)|\]
\[=24.5\]\par
\vspace{1cm}
\textbf{\hspace{1cm}(b)}\par
\vspace{0.5cm}
 Here given :
\[ x_1 = a, y_1 = b+c \]
\[ x_2 = b, y_2 = c+a \]
\[ x_3 = c, y_3 = a+b \]
\[\text{Area} = \frac{1}{2} |(x_1y_2-x_2y_1)+(x_2y_3-x_3y_2)+(x_3y_1-x_1y_3)| \]
\[=\frac{1}{2} |[a(c+a)-b(b+c)]+[b(a+b)-c(c+a)]+[c(b+c)-a(a+b)]|\]
\[=0\]\par
\vspace{1cm}
\textbf{2.}\par
\vspace{0.5cm}
Here given :
\[ x_1 = -1, y_1 = 2 \]
\[ x_2 = 2, y_2 = -1 \]
\[ x_3 = h, y_3 = 3 \]\par

If these three points are collinear , The area made by these points must be zero. So,
\[\text{Area} = \frac{1}{2} |(x_1y_2-x_2y_1)+(x_2y_3-x_3y_2)+(x_3y_1-x_1y_3)| \]
\[=\frac{1}{2} |(1-4)+(6+h)+(2h+3)|\]
\[=3h+6\]\par
Now,
\[3h+6=0\]
\[\Longrightarrow h=-2\]
\[\text{(Showed)}\]\par
\newpage
\textbf{3.}\par
\vspace{0.5cm}
Here given : 
\[\text{A}=(1,2)\]
\[\text{B}=(-5,6)\]
\[\text{C}=(7,-4)\]
\[\text{D}=(k,-2)\]\par
Now,
\[\text{Area} = \frac{1}{2} |(x_1y_2-x_2y_1)+(x_2y_3-x_3y_2)+(x_3y_4-x_4y_3)+(x_4y_1-x_1y_4)| \]
\[=\frac{1}{2}|(6+10)+(20-42)+(-14+4k)+(2k+2)|\]
\[=\frac{1}{2}(6k-18)\]
\[=3k-9\]\par
Now,
\[3k-9=0\]
\[\Longrightarrow k=3\]\par
\vspace{1cm}
\textbf{4.}\par
\vspace{0.5cm}
Here given : 
\[ x_1 = 4, y_1 = 2 \]
\[ x_2 = 7, y_2 = 5 \]
\[ x_3 = 9, y_3 = 7 \]\par

In case these points lie on a straight line, the area made by these points is Zero.\par
\vspace{0.5cm}
Now,
\[\text{Area} = \frac{1}{2} |(x_1y_2-x_2y_1)+(x_2y_3-x_3y_2)+(x_3y_1-x_1y_3)| \]
\[=(20-14)+(49-45)+(18-28)\]
\[=6+4-10\]
\[=0\]
\vspace{0.5cm}
Therefore, these points lie on a straight line.\par
\vspace{1cm}
\textbf{5.}\par
\vspace{0.5cm}
\textbf{Same as Q-3}\par
\newpage
\textbf{6.}\par
\vspace{0.5cm}
Let, 
\[ \text{A}=(1,0) \]
\[ \text{B}=(2,-4) \]
\[ \text{C}=(-5,-2)\]\par
The orthocenter is the intersection point of the altitudes drawn from the vertices of the triangle to the opposite sides.\par
\vspace{0.4cm}
Now, equation to \textbf{BC} is \par
\[\frac{y-y_1}{y_1-y_2}=\frac{x-x_1}{x_1-x_2}\]
\[\Longrightarrow\frac{y-y_1}{y_1-y_2}=\frac{x-x_1}{x_1-x_2}\]
\[\Longrightarrow\frac{y+4}{-4+2}=\frac{x-2}{2+5}\]
\[\Longrightarrow2x+7y+24=0\]
\[....(i)\]\par
Equation of Perpendicular to \textbf{BC} from \textbf{A} is\par
\[7x-2y+k=0\]
\[\text{This line goes through the point } (1,0)\]
\[\text{So,} 7(1)-2(0)+k=0\]
\[\Longrightarrow k=-7\]
\[\text{And the equation becomes, }7x-2y-7=0 \]
\[....(ii)\]\par
Similarly,
Equation of \textbf{AC} is \par
\[4x=y-4=0\]
\[....(iii)\]\par
And Equation of Perpendicular to \textbf{AC} from \textbf{B} is\par
\[x-4y-3=0\]
\[....(iv)\]\par
From equation (ii) and (iv) we get ,\par
\[\text{The intersection point is } = (\frac{11}{13} , \frac{-7}{13})\]\par
\vspace{0.5cm}
This point is the orthocentre of the triangle.\par



\end{flushleft}
\end{document}
