\documentclass{article}
\usepackage{graphicx}
\usepackage{titling}
\usepackage{tikz}
\usepackage{geometry}
\usepackage{amsmath}
\usetikzlibrary{calc}

\geometry{a4paper, margin=1in}

\title{1st assignment}
\author{Maruf Hasan}
\date{\today}

\begin{document}

\begin{titlepage}
    \begin{tikzpicture}[remember picture,overlay]
        \draw[line width = 2pt]
            ($(current page.north west) + (2cm,-2cm)$) rectangle
            ($(current page.south east) + (-2cm,2cm)$);
    \end{tikzpicture}
    
    \begin{center}
        \vspace{2cm}
        \includegraphics[width=0.3\textwidth]{IU logo.jpg}\par
        \vspace{0.5cm}
        {\huge\textbf{Department of CSE}}\par
        \vspace{1cm}
        {\Large\textbf{Course Title:} Geometry and Vector Analysis\par}   
        {\Large\textbf{Course Code:}  MATH-2109\par}
        \vspace{1cm}
        {\huge\textbf{Assignment on Chapter 1 \& 2}\par}
        \vspace{2cm}
        \Large\textbf{Submitted by:}\par
        \textbf{Name:} Maruf Hasan\par
        \textbf{Roll No:} 2114024\par
        \textbf{Reg No:} 1089\par
        \textbf{Session:} 2021-2022\par
        \vspace{2cm}
        \Large\textbf{Submission Date:} October 19, 2024\par
    \end{center}
\end{titlepage}

\newpage

\begin{flushleft}
\textbf{1. Find the areas of triangles whose vertices are}\par
\textbf{(a) $(-3, 4)$, $(6, 2)$, $(4, -3)$}\par
\textbf{(b) $(a, b+c)$, $(b, c+a)$, $(c, a+b)$}\par
\vspace{0.4cm}
\textbf{2. If three points $(-1, 2), (2, -1),(h,3)$ are collinear ; show that $h=-2$}\par
\vspace{0.4cm}
\textbf{3. If the area of quadriatel, whose angular points $A, B, C, D$ taken in order are $(1,2),(-5,6),(7,-4),(k,-2)$ be zero find the value of $k$.}\par
\vspace{0.4cm}
\textbf{4. Show that three points $(4,2),(7,5),(9,4)$ lie on a right line.}\par
\vspace{0.4cm}
\textbf{5. If the area of the quadrilateral formed by the points $(1,3),(2,-5),(6,-2)$ and $(5,k)$ taken in order be $30$ , show that $k=4$ . }\par
\vspace{0.4cm}
\textbf{6. Find the coordinates of the orthocentre of the triangle whose points are $(1,0),(2,-4)$ and $(-5,-2)$ .}\par
\vspace{0.4cm}
\textbf{7. If the points $(a,b),(a',b')$ and $(a-a' , b-b')$ are colinear, show that their joint passes through the origin and that $ab'=ab'$ .}\par
\vspace{0.4cm}
\textbf{8. In what ratio is the straight line joining the pts. $(3,4)$ and $(8,1)$ divides by the $X$ axis. Find the abscissa of this point on $X$ axis.}\par
\vspace{0.4cm}
\textbf{9. Find the polar co-ordinates of the point whose cartesian co-ordinates are}\par
\textbf{(a) $(-\sqrt{3},1)$  }\par
\textbf{(b) $(5,12)$}\par
\vspace{0.4cm}
\textbf{10. Find the cartesian co-ordinates of the points whose polar co-ordinates are}\par
\textbf{(a) $(5, -\frac{\pi}{4})$}\par
\textbf{(b) $(2, 330^{\circ})$}\par
\vspace{0.4cm}
\textbf{11. Find the polar distance between the points whose polar co-ordinates are }\par
\textbf{(a) $(-3, 45^{\circ})$ and $(7, 105^{\circ})$}\par
\textbf{(b) $(\sqrt{2}, \frac{5\pi}{4} )$ and $(2, \frac{2\pi}{3})$}\par
\vspace{0.4cm}
\textbf{12. Find the area of triangle , the polar co-ordinates of whose angular points are $(a , \theta), (2a , \theta+\frac{\pi}{3}), (3a , \theta+\frac{2\pi}{3})$}\par
\vspace{0.4cm}
\textbf{13. Change the equations to polar co-ordinates}\par
\textbf{(a) $(x^2+y^2)^2 = 2a^2xy$}\par
\textbf{(b) $x^4 + x^2y^2 - (x+y)^2 = 0$}\par
\vspace{0.4cm}
\textbf{14. Transform the polar co-ordinates to equations}\par
\textbf{(a) $r^{2}-2r(\cos\theta-\sin\theta)-7 = 0$}\par
\textbf{(b) $r(1- e\cos\theta) = ep$}\par

\newpage
\textbf{1. \hspace{0.4cm}(a)}\par
\vspace{0.5cm}
 Here given : 
\[ x_1 = -3, y_1 = 4 \]
\[ x_2 = 6, y_2 = 2 \]
\[ x_3 = 4, y_3 = -3 \]
\[\text{Area} = \frac{1}{2} |(x_1y_2-x_2y_1)+(x_2y_3-x_3y_2)+(x_3y_1-x_1y_3)| \]
\[=\frac{1}{2} |(-6-24)+(-18-8)+(16-9)|\]
\[=24.5\]\par
\vspace{1cm}
\textbf{\hspace{1cm}(b)}\par
\vspace{0.5cm}
 Here given :
\[ x_1 = a, y_1 = b+c \]
\[ x_2 = b, y_2 = c+a \]
\[ x_3 = c, y_3 = a+b \]
\[\text{Area} = \frac{1}{2} |(x_1y_2-x_2y_1)+(x_2y_3-x_3y_2)+(x_3y_1-x_1y_3)| \]
\[=\frac{1}{2} |[a(c+a)-b(b+c)]+[b(a+b)-c(c+a)]+[c(b+c)-a(a+b)]|\]
\[=0\]\par
\vspace{1cm}

\textbf{2.}\par
\vspace{0.5cm}
Here given :
\[ x_1 = -1, y_1 = 2 \]
\[ x_2 = 2, y_2 = -1 \]
\[ x_3 = h, y_3 = 3 \]\par

If these three points are collinear , The area made by these points must be zero. So,
\[\text{Area} = \frac{1}{2} |(x_1y_2-x_2y_1)+(x_2y_3-x_3y_2)+(x_3y_1-x_1y_3)| \]
\[=\frac{1}{2} |(1-4)+(6+h)+(2h+3)|\]
\[=3h+6\]\par
Now,
\[3h+6=0\]
\[\Longrightarrow h=-2\]
\[\text{(Showed)}\]\par
\newpage

\textbf{3.}\par
\vspace{0.5cm}
Here given : 
\[\text{A}=(1,2)\]
\[\text{B}=(-5,6)\]
\[\text{C}=(7,-4)\]
\[\text{D}=(k,-2)\]\par
Now,
\[\text{Area} = \frac{1}{2} |(x_1y_2-x_2y_1)+(x_2y_3-x_3y_2)+(x_3y_4-x_4y_3)+(x_4y_1-x_1y_4)| \]
\[=\frac{1}{2}|(6+10)+(20-42)+(-14+4k)+(2k+2)|\]
\[=\frac{1}{2}(6k-18)\]
\[=3k-9\]\par
Now,
\[3k-9=0\]
\[\Longrightarrow k=3\]\par
\vspace{1cm}

\textbf{4.}\par
\vspace{0.5cm}
Here given : 
\[ x_1 = 4, y_1 = 2 \]
\[ x_2 = 7, y_2 = 5 \]
\[ x_3 = 9, y_3 = 7 \]\par

In case these points lie on a straight line, the area made by these points is Zero.\par
\vspace{0.5cm}
Now,
\[\text{Area} = \frac{1}{2} |(x_1y_2-x_2y_1)+(x_2y_3-x_3y_2)+(x_3y_1-x_1y_3)| \]
\[=(20-14)+(49-45)+(18-28)\]
\[=6+4-10\]
\[=0\]
\vspace{0.5cm}
Therefore, these points lie on a straight line.\par
\vspace{1cm}

\textbf{5.}\par
\vspace{0.5cm}
Here given : 
\[\text{A}=(1,3)\]
\[\text{B}=(2,-5)\]
\[\text{C}=(6,-2)\]
\[\text{D}=(5,k)\]\par
Now,
\[\text{Area} = \frac{1}{2} |(x_1y_2-x_2y_1)+(x_2y_3-x_3y_2)+(x_3y_4-x_4y_3)+(x_4y_1-x_1y_4)| \]
\[=\frac{1}{2}|(-5-6)+(-4+30)+(6k+10)+(15-k)|\]
\[=\frac{1}{2}(5k+40)\]\par
Now,
\[=\frac{1}{2}(5k+40)=30\]
\[\Longrightarrow k=4\]\par
\[\text(showed)\]\par
\vspace{1cm}

\textbf{6.}\par
\vspace{0.5cm}
Let, 
\[ \text{A}=(1,0) \]
\[ \text{B}=(2,-4) \]
\[ \text{C}=(-5,-2)\]\par
The orthocenter is the intersection point of the altitudes drawn from the vertices of the triangle to the opposite sides.\par
\vspace{0.4cm}
Now, equation to \textbf{BC} is \par
\[\frac{y-y_1}{y_1-y_2}=\frac{x-x_1}{x_1-x_2}\]
\[\Longrightarrow\frac{y-y_1}{y_1-y_2}=\frac{x-x_1}{x_1-x_2}\]
\[\Longrightarrow\frac{y+4}{-4+2}=\frac{x-2}{2+5}\]
\[\Longrightarrow2x+7y+24=0\]
\[....(i)\]\par
Equation of Perpendicular to \textbf{BC} from \textbf{A} is\par
\[7x-2y+k=0\]
\[\text{This line goes through the point } (1,0)\]
\[\text{So,} 7(1)-2(0)+k=0\]
\[\Longrightarrow k=-7\]
\[\text{And the equation becomes, }7x-2y-7=0 \]
\[....(ii)\]\par
Similarly,
Equation of \textbf{AC} is \par
\[4x=y-4=0\]
\[....(iii)\]\par
And Equation of Perpendicular to \textbf{AC} from \textbf{B} is\par
\[x-4y-3=0\]
\[....(iv)\]\par
From equation (ii) and (iv) we get ,\par
\[\text{The intersection point is } = (\frac{11}{13} , \frac{-7}{13})\]\par
\vspace{0.5cm}
This point is the orthocentre of the triangle.\par
\newpage

\textbf{7.}\par
\vspace{0.5cm}
Here three points are
\[(a , b )\]
\[(a^{'} , b^{'})\]
\[\text{and }(a-a^{'},b-b^{'})\]
If three points are collinear, they lie on the same line.\par
\vspace{0.4cm}
The join of $(a , b)$ and $(a^{'},b^{'})$ is \par
\[ \frac{x-a}{a-a^{'}} = \frac{y-b}{b-b^{'}}\]\par
Since it passes through $(a-a^{'} , b-b^{'})$\par
\[\Longrightarrow\frac{a-a^{'}-a}{a-a^{'}}=\frac{b-b^{'}-b}{b-b^{'}}\] \par
\[\Longrightarrow ab^{'} = a{'}b\]
so, $x (b-b^{'}) = y (a-a^{'})$ , which passes through the origin.\par
\vspace{1cm}

\textbf{8.}\par
\vspace{0.5cm}
Let, the line joining the points $(3, 4)$ and $(8, 1)$ divided by the x-axis at $(x_1,0)$ in the ratio $(k:1)$ .\par
\[\text{so, } \frac{k.y_2 + 1 . y_1}{k+1} = 0\]
\[\Longrightarrow \frac{k.1 + 4}{k+1} = 0\]
\[\Longrightarrow k+4 = 0\]
\[\Longrightarrow k = -4\]\par
That means the required ratio is $-4 : 1$ .\par
\vspace{0.4cm}
Now, \[x=\frac{k.x_2 + 1.x_1}{k+1}\]
\[\Longrightarrow x=\frac{-4.8 + 3}{-4+1}\]
\[\Longrightarrow x = \frac{29}{3}\]\par
\vspace{0.4cm}
This is the abscissa of this point on X axis.\par
\newpage

\textbf{9. \hspace{0.4cm}(a)}\par
\vspace{0.5cm}
Here the given cartesian co-ordinate is $(-\sqrt{3},1)$\par
Now,\[r=\sqrt{x^{2}+y^{2}}\]
\[\Longrightarrow r = \sqrt{({\sqrt{3}})^{2} + 1^{2}}\]
\[\Longrightarrow r = 2\]
And,
\[\tan \theta = \frac{y}{x}\]
\[\Longrightarrow \tan \theta = \frac{1}{-\sqrt{3}}\]
\[\Longrightarrow \theta = \frac{5\pi}{6}\]
So, the polar co-ordinate is $(2,\frac{5\pi}{6})$ .\par
\vspace{1cm}
\textbf{\hspace{1cm}(b)}\par
\vspace{0.5cm}
Here the given cartesian co-ordinate is $(5,12)$\par
Now,\[r=\sqrt{x^{2}+y^{2}}\]
\[\Longrightarrow r = \sqrt{5^{2} + 12^{2}}\]
\[\Longrightarrow r = 13\]
And,
\[\tan \theta = \frac{y}{x}\]
\[\Longrightarrow \tan \theta = \frac{12}{5}\]
\[\Longrightarrow \theta = \tan^{-}(\frac{12}{5})\]
So, the polar co-ordinate is $(13,\tan^{-}(\frac{12}{5}))$ .\par
\vspace{1cm}

\textbf{10.\hspace{0.4cm}(a)}\par
\vspace{0.5cm}
Here the given polar co-ordinate is $(5,-\frac{\pi}{4})$\par
Now,\par
\[x=5 \cos{(-\frac{\pi}{4})}\]
\[\Longrightarrow x = \frac{5}{\sqrt{2}}\]\par
And,\par
\[y=5\sin{(-\frac{\pi}{4})}\]
\[\Longrightarrow y = -\frac{5}{\sqrt{2}}\]\par
The cartesin co-ordinates are $(\frac{5}{\sqrt{2}},-\frac{5}{\sqrt{2}})$.\par
\newpage
\textbf{\hspace{1cm}(b)}\par
Here the given polar co-ordinate is $(2, 330^{\circ})$\par
Now,\par
\[x=2 \cos{(330^{\circ})}\]
\[\Longrightarrow x = 2\cos{(360^{\circ}-30^{\circ}})\]\par
\[\Longrightarrow x = 2 \cos{(30^{\circ})}\]
\[\Longrightarrow x = \sqrt{3}\]
And,\par
\[y=2 \sin{(330^{\circ})}\]
\[\Longrightarrow y = 2\sin{(360^{\circ}-30^{\circ}})\]\par
\[\Longrightarrow y = -2 \sin{(30^{\circ})}\]
\[\Longrightarrow y = -1 \]
The cartesin co-ordinates are $(\sqrt{3},-1)$.\par
\vspace{1cm}

\textbf{11. \hspace{0.4cm}(a)}\par
\vspace{0.5cm}
Given polar co-ordinates are\par
\[(-3, 45^{\circ}) \text { and } (7, 105^{\circ})\]
Now,\par
\[\text{Distance }= \sqrt{(r_1)^{2}+(r_2)^{2}-2r_1 r_2 \cos{(\theta_2 -\theta_1)}}\]
\[=\sqrt{(-3)^{2}+(7)^{2}-2.(-3).7\cos{(105^{\circ}-45^{\circ})}}\]
\[=\sqrt{79}\]\par
\vspace{1cm}
\textbf{\hspace{1cm}(b)}\par
\vspace{0.5cm}
Given polar co-ordinates are\par
\[(\sqrt{2}, \frac{5\pi}{4} ) \text { and } (2, \frac{2\pi}{3})\]
Now,\par
\[\text{Distance }= \sqrt{(r_1)^{2}+(r_2)^{2}-2r_1 r_2 \cos{(\theta_2 -\theta_1)}}\]
\[=\sqrt{(\sqrt{2})^{2}+(2)^{2}-2.(\sqrt{2}).2\cos{(\frac{5\pi}{4}-\frac{2\pi}{3})}}\]
\[=\sqrt{6-4\sqrt{2}\cos{(\frac{7\pi}{12})}}\]
\[=\sqrt{6-4\sqrt{2}(-\frac{\sqrt{3}+1}{2\sqrt{2}})}\]
\[=\sqrt{(\sqrt{3}+1)^{2}}\]
\[=1+\sqrt{3}\]\par
\newpage

\textbf{12. }\par
\vspace{0.5cm}
Let,
\[\text{A}=(a , \theta)\]
\[\text{B}=(2a , \theta+\frac{\pi}{3})\]
\[\text{C}=(3a , \theta+\frac{2\pi}{3})\]\par
Now,\par
\[\text{$\triangle$ ABC }=\frac{1}{2}|r_1 r_2 sin (\theta_2 -\theta_1)+r_2 r_3 sin (\theta_3 -\theta_2)+r_3 r_1 sin (\theta_1 -\theta_3)|\]
\[=\frac{1}{2}|2a^{2}\sin{(\frac{\pi}{3})}+6a^{2}\sin{(\frac{\pi}{3})}+3a^{2}\sin{(-\frac{2\pi}{3})}|\]
\[=\frac{1}{2}(2a^{2}\frac{\sqrt{3}}{2}+6a^{2}\frac{\sqrt{3}}{2}-3a^{2}\frac{\sqrt{3}}{2})\]
\[=\frac{5\sqrt{3}}{4}a^{2}\]\par
\vspace{1cm}

\textbf{13 \hspace{0.4cm}(a)}\par
\vspace{0.5cm}
Given equation\par
\[(x^{2}+y^{2})^{2}=2a^{2}xy\]\par
Putting $x = r \cos \theta $ and $y = r \sin \theta$ we get,\par
\[((r \cos^{2}\theta)+(r \sin^{2}\theta))^{2} = 2 a^{2}. r \cos{\theta} . r \sin{\theta}\]
\[\Longrightarrow (r^{2}(\cos^{2}{\theta} + \sin^{2}{\theta}))^{2} = 2a^{2}r^{2} \sin{\theta}\cos{\theta}\]
\[\Longrightarrow r^{4} = 2 a^{2}r^{2} \sin{\theta}\cos{\theta}\]
\[\Longrightarrow r^{2} = a^{2} \sin{2\theta}\]\par
\vspace{1cm}
\textbf{\hspace{1cm}(b)}\par
\vspace{0.5cm}
Given equation\par
\[x^4 + x^2y^2 - (x+y)^2 = 0\]
Putting $x = r \cos \theta $ and $y = r \sin \theta$ we get,\par
\[r^{2} \cos^{2}\theta r^{2} - (r^{2}+2r^{2} \cos\theta \sin\theta) = 0\]
\[\Longrightarrow r^{2} \cos^{2}\theta = 1 + \sin{2\theta}\]
\[\Longrightarrow r^{2}=\frac{(\cos\theta + \sin \theta)^{2}}{\cos^{2}\theta}\]
\[\Longrightarrow r = \pm \frac{\cos \theta + \sin \theta}{\cos \theta}\]
\[\Longrightarrow r = \pm \tan{\theta}\]\par
\newpage

\textbf{14.\hspace{0.4cm}(a)}\par
\vspace{0.5 cm}
Here given,\par
\[r^{2}-2r(\cos\theta-\sin\theta)-7 = 0\]
\[\Longrightarrow r^{2}-2(r\cos\theta -r\sin\theta)-7=0\]
Putting $r \cos \theta = x $ and $r \sin \theta = y $ we get,\par
\[x^{2}+y^{2}-2(x-y)-7=0\]
\[\Longrightarrow x^{2}+y^{2}-2x+2y-7=0\]\par
\vspace{1cm}
\textbf{\hspace{1cm}(b)}\par
\vspace{0.5cm}
Here given,\par
\[r(1- e\cos\theta) = ep\]
\[\Longrightarrow r - e\hspace{0.1cm} r\cos \theta = ep\]
Putting $r \cos \theta = x $ and $r \sin \theta = y $ we get,\par
\[r - ex = ep\]
\[\Longrightarrow r = e (x+p)\]
\[\Longrightarrow r^{2} = e^{2}(x+p)^{2}\]
\[\Longrightarrow x^{2}+y^{2} = e^{2} (x+p)^{2}\]\par
\vspace{8cm}
\[\textbf{END}\]


\end{flushleft}
\end{document}
